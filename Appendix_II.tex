\section{Appendix II}

First of all, we recall the definitions:
\begin{equation}
    \boldsymbol{J}_{i,j}(\boldsymbol{k}) = \sum_{\boldsymbol{d}}\boldsymbol{J}_{i,j}(\boldsymbol{d})e^{-i\boldsymbol{k}\boldsymbol{d}}
\end{equation}
\begin{equation}
    A(\boldsymbol{k})^{i,j} = \dfrac{\sqrt{S_i, S_j}}{2}\boldsymbol{u}^T_i\boldsymbol{J}_{i,j}(-\boldsymbol{k})\overline{\boldsymbol{u}}_j,
\end{equation}
\begin{equation}
    B(\boldsymbol{k})^{i,j} = \dfrac{\sqrt{S_i, S_j}}{2}\boldsymbol{u}^T_i\boldsymbol{J}_{i,j}(-\boldsymbol{k})\boldsymbol{u}_j,
\end{equation}
\begin{equation}
    C(\boldsymbol{k})^{i,j} = C^{i,j} = \delta_{i,j}\sum_{l}S_l \boldsymbol{v}^T_i\boldsymbol{J}_{i, l}(\boldsymbol{0})\boldsymbol{v}_l.
\end{equation}

There are three statements, that requires proofs:
\begin{equation}
    A^{i,j}(\boldsymbol{k}) = \overline{A^{j,i}(\boldsymbol{k})}
\end{equation}
\begin{equation}
    B^{i,j}(\boldsymbol{k}) = B^{j,i}(-\boldsymbol{k})
\end{equation}
\begin{equation}
    C^{i,j} = C^{j,i}
\end{equation}

Symmetry of the exchange between two sites (note, that transposition and the switch of indices $i,j$ are two different operations):
\begin{equation}
    \boldsymbol{J}_{i,j}(\boldsymbol{d}) = \boldsymbol{J}_{j,i}^T(-\boldsymbol{d})
\end{equation}
which leads to the:
\begin{equation}
    \boldsymbol{J}_{i,j}(\boldsymbol{k}) = \boldsymbol{J}_{j,i}^T(-\boldsymbol{k})
\end{equation}

Then:
\begin{equation}
    \overline{A^{j,i}(\boldsymbol{k})} = \dfrac{\sqrt{S_i, S_j}}{2}\overline{\boldsymbol{u}}^T_j\boldsymbol{J}_{j,i}(\boldsymbol{k})\boldsymbol{u}_i = 
    \dfrac{\sqrt{S_i, S_j}}{2}\overline{\boldsymbol{u}}^T_j\boldsymbol{J}_{i,j}^T(-\boldsymbol{k})\boldsymbol{u}_i
\end{equation}
since $\overline{A^{j,i}(\boldsymbol{k})}$ is a complex number, we can transpose it, without modification of the result:
\begin{equation}
    \overline{A^{j,i}(\boldsymbol{k})} =  (\dfrac{\sqrt{S_i, S_j}}{2}\overline{\boldsymbol{u}}^T_j\boldsymbol{J}_{i,j}^T(-\boldsymbol{k})\boldsymbol{u}_i)^T =
    \dfrac{\sqrt{S_i, S_j}}{2}\boldsymbol{u}^T_i\boldsymbol{J}_{i,j}(-\boldsymbol{k})\overline{\boldsymbol{u}}_j = A(\boldsymbol{k})^{i,j}
\end{equation}

Second one:
\begin{equation}
    B^{j,i}(-\boldsymbol{k}) = \dfrac{\sqrt{S_i, S_j}}{2}\boldsymbol{u}^T_j\boldsymbol{J}_{j,i}(\boldsymbol{k})\boldsymbol{u}_i =
    \dfrac{\sqrt{S_i, S_j}}{2}\boldsymbol{u}^T_j\boldsymbol{J}_{i,j}^T(-\boldsymbol{k})\boldsymbol{u}_i
\end{equation}
since $B^{j,i}(-\boldsymbol{k})$ is a complex number, we can transpose it, without modification of the result:
\begin{equation}
    B^{j,i}(-\boldsymbol{k}) =  (\dfrac{\sqrt{S_i, S_j}}{2}\boldsymbol{u}^T_j\boldsymbol{J}_{i,j}^T(-\boldsymbol{k})\boldsymbol{u}_i)^T =
    \dfrac{\sqrt{S_i, S_j}}{2}\boldsymbol{u}^T_i\boldsymbol{J}_{i,j}(-\boldsymbol{k})\boldsymbol{u}_j = B(\boldsymbol{k})^{i,j}
\end{equation}

$C^{i,j} = C^{j,i}$ because of the kronecker delta in the definition.
