\section{Appendix I}

In this section The magnon dispersion law is derived from the Hamiltonian in eq.~\eqref{eq:hh-main}.
First of all, the Hamiltonian is rewritten with the raising and lowering spin operators:

\begin{equation}
    \hat{S}_i^{\pm} = \hat{S}_i^x \pm \iu \hat{S}_i^y
\end{equation}

\begin{equation}
    \begin{matrix}
        \hat{\mathbf{S}}_i^T \hat{\mathbf{S}}_j = 
        \hat{S}_i^x \hat{S}_j^x + \hat{S}_i^y \hat{S}_j^y + \hat{S}_i^z \hat{S}_j^z; &
        \hat{S}_i^x \hat{S}_j^x + \hat{S}_i^y \hat{S}_j^y = 
        \dfrac{1}{2}\left(\hat{S}_i^+\hat{S}_j^- + \hat{S}_i^-\hat{S}_j^+\right)
    \end{matrix}
\end{equation}

\begin{equation}
    \hat{H} = -J \sum_{ij} \left(\dfrac{1}{2}\left(
        \hat{S}_i^+\hat{S}_j^- + \hat{S}_i^-\hat{S}_j^+\right) + \hat{S}_i^z \hat{S}_j^z\right)
\end{equation}

Since the commutator is 
\begin{equation}
    \left[\hat{S}_i^+\hat{S}_j^-\right] = 2\hat{S}_i^z\delta_{ij}
\end{equation}
and $i\ne j$ in the sum the Hamiltonian becomes:
\begin{equation}
    \hat{H} =-J \sum_{ij} \left(\dfrac{1}{2}\left(
            \hat{S}_j^-\hat{S}_i^+ + \hat{S}_i^-\hat{S}_j^+\right) + \hat{S}_i^z \hat{S}_j^z\right)
\end{equation}

The spin-wave Hamiltonian is obtained with the linearised Holstein–Primakoff formalism.

\begin{equation}
    \begin{matrix}
        \hat{S}_i^+ = \sqrt{2S}\hat{a}_i \\
        \hat{S}_i^- = \sqrt{2S}\hat{a}_i^{\dag} \\
        \hat{S}_i^z = S - \hat{a}_i^{\dag}\hat{a}_i
    \end{matrix}
\end{equation}

\begin{equation}
    \hat{H} = -J \sum_{ij} \left(\dfrac{1}{2}\left(
        2S\hat{a}_j^{\dag}\hat{a}_i + 2S\hat{a}_i^{\dag}\hat{a}_j\right) + 
        \left(S - \hat{a}_i^{\dag}\hat{a}_i\right)\left(S - \hat{a}_j^{\dag}\hat{a}_j\right)\right)
\end{equation}

\begin{equation}
    \hat{H} = E_0 + \hat{H}^{(2)} + \dots 
\end{equation}
\begin{equation}
    E_0 = -JS^2Nn \label{eq:zero-energy}
\end{equation}
\begin{equation}
    \hat{H}^{(2)} = -JS \sum_{ij} \left(\hat{a}_j^{\dag}\hat{a}_i + \hat{a}_i^{\dag}\hat{a}_j - 
    \hat{a}_i^{\dag}\hat{a}_i - \hat{a}_j^{\dag}\hat{a}_j\right)
    \label{eq:quadratic-ham}
\end{equation}
where $N$ is the number of spins in the system, $n$ - number of neighbors for each spin ($6$ in the case of cubic system).
From this point the quadratic part of the Hamiltonian $\hat{H}^{(2)}$ is considered.

The Fourier transform is introduced to move from the local operators $\hat{a}_i^{\dag}$ and $\hat{a}_i$
to the collective creation and annihilation operators $\hat{a}_k^{\dag}$ and $\hat{a}_k$:
\begin{equation}
    \hat{a}_i = \dfrac{1}{\sqrt{N}}\sum_k e^{\iu\mathbf{k}\mathbf{r}_i} \hat{a}_k 
\end{equation}
\begin{equation}
    \hat{a}_i^{\dag} = \dfrac{1}{\sqrt{N}}\sum_k e^{-\iu\mathbf{k}\mathbf{r}_i} \hat{a}_k^{\dag} 
\end{equation}
\begin{equation}
    \dfrac{1}{N}\sum_i e^{\iu(\mathbf{k} - \mathbf{k}^{\prime})\mathbf{r}_i} = \delta_{kk^{\prime}} 
\end{equation}

\begin{equation}
\begin{aligned}
    \hat{H}^{(2)}  = -JS \sum_i\sum_j \dfrac{1}{N}\left[
        \left(\sum_k e^{-\iu\mathbf{k}\mathbf{r}_j} \hat{a}_k^{\dag}\right)
        \left(\sum_{k^{\prime}} e^{\iu\mathbf{k^{\prime}}\mathbf{r}_i} \hat{a}_{k^{\prime}}\right)\right. \\
        +
        \left(\sum_k e^{-\iu\mathbf{k}\mathbf{r}_i} \hat{a}_k^{\dag}\right)
        \left(\sum_{k^{\prime}} e^{\iu\mathbf{k^{\prime}}\mathbf{r}_j} \hat{a}_{k^{\prime}}\right)  \\
        -
        \left(\sum_k e^{-\iu\mathbf{k}\mathbf{r}_i} \hat{a}_k^{\dag}\right)
        \left(\sum_{k^{\prime}} e^{\iu\mathbf{k^{\prime}}\mathbf{r}_i} \hat{a}_{k^{\prime}}\right)  \\
        -
        \left.\left(\sum_k e^{-\iu\mathbf{k}\mathbf{r}_j} \hat{a}_k^{\dag}\right)
        \left(\sum_{k^{\prime}} e^{\iu\mathbf{k^{\prime}}\mathbf{r}_j} \hat{a}_{k^{\prime}}\right) \right]
\end{aligned}
\end{equation}

Since for each $i$ there is the same pattern of neighbors sum over $j$ does not depend on $i$ and it can be moved freely.
Lets define $\boldsymbol{\delta}_j = \mathbf{r}_j - \mathbf{r}_i$ and rewrite the equation:

\begin{equation}
\begin{aligned}
    \hat{H}^{(2)}  = -JS \sum_k\sum_{k^{\prime}}\sum_j \left[
        e^{-\iu\boldsymbol{\delta}_j\mathbf{k}}
        \left(\dfrac{1}{N}\sum_ie^{\iu(\mathbf{k^{\prime}}-\mathbf{k})\mathbf{r}_i}\right) 
        \hat{a}_k^{\dag}\hat{a}_{k^{\prime}}\right. \\
        +
        e^{\iu\boldsymbol{\delta}_j\mathbf{k}}
        \left(\dfrac{1}{N}\sum_ie^{\iu(\mathbf{k^{\prime}}-\mathbf{k})\mathbf{r}_i}\right)
         \hat{a}_k^{\dag}\hat{a}_{k^{\prime}}  \\
        -
        \left(\dfrac{1}{N}\sum_ie^{\iu(\mathbf{k^{\prime}}-\mathbf{k})\mathbf{r}_i}\right)
        \hat{a}_k^{\dag}\hat{a}_{k^{\prime}}  \\
        -
        \left.e^{\iu(\mathbf{k^{\prime}}-\mathbf{k})\boldsymbol{\delta}_j}
        \left(\dfrac{1}{N}\sum_ie^{\iu(\mathbf{k^{\prime}}-\mathbf{k})\mathbf{r}_i}\right)
        \hat{a}_k^{\dag}\hat{a}_{k^{\prime}} \right]
\end{aligned}
\end{equation}
every equation in round parenthesis is equal to $\delta_{kk^{\prime}}$ and the Hamiltonian becomes

\begin{multline}
    \hat{H}^{(2)} = -JS\sum_k\sum_j\left[e^{-\iu\boldsymbol{\delta}_j\mathbf{k}}
    \hat{a}_k^{\dag}\hat{a}_k
    +
    e^{\iu\boldsymbol{\delta}_j\mathbf{k}}
     \hat{a}_k^{\dag}\hat{a}_k 
    -
    \hat{a}_k^{\dag}\hat{a}_k 
    -
    e^{\iu(\mathbf{k}-\mathbf{k})\boldsymbol{\delta}_j}
    \hat{a}_k^{\dag}\hat{a}_k\right] \\
     = 2JS\sum_k\sum_j(1 - \cos(\boldsymbol{\delta}_j\mathbf{k}))\hat{a}_k^{\dag}\hat{a}_k
\end{multline}
$j$ runs from $1$ to $n$, therefore:
\begin{equation}
    \hat{H}^{(2)} = 2JSn\sum_k\left(1 - \dfrac{1}{n}\sum_j
    \cos(\boldsymbol{\delta}_j\mathbf{k})\right)\hat{a}_k^{\dag}\hat{a}_k = 
    \sum_k \hbar\omega(\mathbf{k})\hat{a}_k^{\dag}\hat{a}_k
\end{equation}

For the cubic system ($l$ - length of the lattice vector):
\begin{align}
    \dfrac{1}{n}\sum_j
    e^{\iu\boldsymbol{\delta}_j\mathbf{k}}  = \dfrac{1}{6}\left(\cos(k_xl) + \cos(-k_xl) + \cos(k_yl) + \cos(-k_yl) + \cos(k_zl) + \cos(-k_zl)\right) \\
    = \dfrac{1}{3}\left(\cos(k_x l) + \cos(k_y l) + \cos(k_z l)\right)
\end{align}
and the final formula for the magnon dispersion is
\begin{equation}
    \hbar\omega(\mathbf{k}) = 2JSn\left(1 - 
    \dfrac{1}{3}\left(\cos(k_x l) + \cos(k_y l) + \cos(k_z l)\right)\right)
\end{equation}