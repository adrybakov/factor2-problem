\documentclass[a4paper,12pt]{article}

\usepackage[utf8]{inputenc}
\usepackage[english]{babel}
\usepackage[T1]{fontenc} % for correct << and >>
\usepackage{amssymb, amsmath, multicol, amsthm, mathtools}
\usepackage{csquotes}
\usepackage{mathrsfs}
\usepackage{graphicx}
\usepackage{multirow}
\usepackage{caption}
\usepackage{subcaption}
\usepackage{indentfirst}
\usepackage{esvect}
\usepackage{float} 
\usepackage[version=4]{mhchem}
\mathtoolsset{showonlyrefs=true}
\usepackage{hyperref}
\usepackage[rgb,table,xcdraw]{xcolor}
\hypersetup{				
	unicode=true,        
	colorlinks=true,       	
	linkcolor=black,        
	citecolor=black,        
	filecolor=magenta,      
	urlcolor=black         
}
\usepackage{pythonhighlight}

\usepackage[left=2cm,right=2cm,
    top=2cm,bottom=2cm]{geometry}
\usepackage{fancyhdr}


\graphicspath{{images}}

\newcommand{\angstrom}{\text{\normalfont\AA}}
\newcommand{\iu}{\mathrm{i}\mkern1mu}
\newcommand*{\hatH}{\hat{\mathcal{H}}}





\begin{document}

    \section{Definition of the Hamiltonian}

        There is a number of details in the definition of the Heisenberg Hamiltonians, which may cause an incompatibility of the direct results. 
        In this paper the main definition is as follows:

        \begin{equation}
            \hat{H} = -J \sum_{ij} \hat{\mathbf{S}}_i^T \hat{\mathbf{S}}_j,
            \label{eq:hh-main}
        \end{equation}
        where $J$ is an isotropic exchange parameter. The double counting is present in the Hamiltonian, i.e. both terms $i\rightarrow j$ and $j \rightarrow i$ are present in the sum. 
        $\hat{\mathbf{S}}_i$ is a $3\times1$ column vector of the spin operators $(\hat{\mathbf{S}}_i^x, \hat{\mathbf{S}}_i^y, \hat{\mathbf{S}}_i^z)$. 
        Index $i$ run over all $N$ sites in the system and index $j$ runs over neighbors for the site $i$. 

        Bold mathematical symbols in this paper represent vectors or matrices and usual symbols~-- scalars. For instance, $J$ is a scalar exchange parameter, while $\mathbf{J}$ is a matrix of exchange, 
        for the isotropic case it is defined as

        \begin{equation}
            \mathbf{J} =
            \begin{pmatrix}
                J & 0 & 0 \\
                0 & J & 0 \\
                0 & 0 & J
            \end{pmatrix}
        \end{equation}

        Several comparisons of the exchange Hamiltonians and consecutive spin wave Hamiltonians are done in this paper 
        and for each one the details of the convergence are discussed. 
        The results are present in both ways: the original source and in the definition of this paper, when possible.

    \section{Ferromagnetic cubic system}

        Parameters and they values for the case study system~-- 
        cubic lattice of ferromagnetic spins oriented along the direction of $z$ axis are defined in this section.

        Lattice (see~Fig.~\ref{fig:lattice}) in cartesian coordinate system is defined by the lattice parameters and angles:
        \begin{equation}
            \begin{matrix}
                \mathbf{a} = (l, 0, 0) & \mathbf{b} = (0, l, 0) & \mathbf{c} = (0, 0, l) \\
                \alpha = 90^{\circ} & \beta = 90^{\circ} & \gamma = 90^{\circ} \\
            \end{matrix}
        \end{equation}
        In each unit cell one spin $\mathbf{S}$ at the position $(0, 0, 0)$ (in relative coordinates) is oriented along z axis.
        Each spin has 6 neighbors as shown in Fig.~\ref{fig:lattice-neighbors} for the $(0, 0, 0)$ unit cell.

        \begin{figure}[H]
            \centering
            \begin{subfigure}[b]{0.49\textwidth}
                \centering
                \includegraphics[height=6cm]{lattice.pdf}
                \caption{}
            \label{fig:lattice}
            \end{subfigure}
            \hfill
            \begin{subfigure}[b]{0.49\textwidth}
                \centering
                \includegraphics[height=6cm]{lattice-neighbors.pdf}
            \caption{}
            \label{fig:lattice-neighbors}
            \end{subfigure}
            \hfill
            \caption{(a) Lattice and (b) 6 neighbors for the spin in $(0, 0, 0)$ unit cell.}
            \label{fig:lattice-both}
        \end{figure}


        The reciprocal lattice is defined as:
        \begin{equation}
            \begin{matrix}
                \mathbf{b_1} = (\dfrac{2\pi}{l}, 0, 0) & \mathbf{b_2} = (0,\dfrac{2\pi}{l}, 0) & \mathbf{b_3} = (0, 0, \dfrac{2\pi}{l}) \\
                k_{\alpha} = 90^{\circ} & k_{\beta} = 90^{\circ} & k_{\gamma} = 90^{\circ} \\
            \end{matrix}
        \end{equation}

        Y-$\Gamma$-X-M-$\Gamma$-R-X-M-R path (see~Fig.~\ref{fig:path}) is used for magnon dispersion plots.

        \begin{figure}[H]
            \centering
            \begin{subfigure}[b]{0.8\textwidth}
                \centering
                \includegraphics[height=10cm]{path.pdf}
            \end{subfigure}
            \hfill
            \caption{Path for the magnon dispersion plots in reciprocal space: Y-$\Gamma$-X-M-$\Gamma$-R-X-M-R.}
            \label{fig:path}
        \end{figure}


        For the final results and for the Figs.~\ref{fig:lattice-both}~and~\ref{fig:path} the following numerical values are used:
        \begin{equation}
            \begin{matrix}
                J = 1 \text{ meV}, & S = 1, & l = 1, & n = 6
            \end{matrix}
        \end{equation}

        For the 1D chain the values are the same, but $n = 2$.

    \section{Main magnon dispersion}

        In Appendix~I magnon dispersion is derived from the Hamiltonian in eq.~\eqref{eq:hh-main}.
        The final result for the cubic system is present in eq.~\ref{eq:main-dispersion} ($n = 6$). 

        \begin{equation}
            \hbar\omega(\mathbf{k}) = 2JSn\left(1 - 
            \dfrac{1}{3}\left(\cos(k_x l) + \cos(k_y l) + \cos(k_z l)\right)\right)
            \label{eq:main-dispersion}
        \end{equation}

        For the 1D chain dispersion law is ($n = 2$)
        \begin{equation}
            \hbar\omega(\mathbf{k}) = 2JSn\left(1 - \left(\cos(k_x l)\right)\right)
            \label{eq:main-dispersion-1D}
        \end{equation}

        It is plotted in Fig.~\ref{fig:main-dispersion} for the path Y-$\Gamma$-X-M-$\Gamma$-R-X-M-R. 
        The picture is produced with the script <<codes/dispersion.py>> using 
        <<main\_dispersion>> function.


        \begin{figure}[H]
            \centering
            \begin{subfigure}[b]{0.8\textwidth}
                \centering
                \includegraphics[height=7cm]{main_dispersion.pdf}
            \end{subfigure}
            \hfill
            \caption{Magnon dispersion plotted with equation~\eqref{eq:main-dispersion}.}
            \label{fig:main-dispersion}
        \end{figure}

    \section{Literature check}
        In this section the result from eq.~\eqref{eq:main-dispersion} is compared with the textbook results. 
        Hamiltonian definitions are compared with the equation~\eqref{eq:hh-main} and conversions are made if necessary, for each source. 
        Detailed analysis of each source is provided in Appendix~II. In Table~\ref{tab:literature-review} the summary of textbook's review is provided.

        \begin{table}[H]
            \centering
            \caption{Comparison of magnon dispersion formulas with textbooks (converted to the notation of this paper).}
            \label{tab:literature-review}
            \def\arraystretch{2.5}
            \begin{tabular}{|c|c|}
                \hline
                Source                             & Formula                                                                                                        \\ \hline
                This paper                         & $\hbar\omega(\mathbf{k}) = 2JSn\left(1 - \dfrac{1}{3}\left(\cos(k_xl) + \cos(k_yl) + \cos(k_zl)\right)\right)$ \\ \hline
                \cite{rezende2020fundamentals}     & $\hbar\omega(\mathbf{k}) = 2nJS\left(1 - \dfrac{1}{3}\left(\cos(k_xl) + \cos(k_yl) + \cos(k_zl)\right)\right)$ \\ \hline
                \cite{blundell2003magnetism}       & $\hbar\omega(\mathbf{k}) = 2nJS\left(1 - \dfrac{1}{3}\left(\cos(k_xl) + \cos(k_yl) + \cos(k_zl)\right)\right)$ \\ \hline
                \cite{gurevich1996magnetization}   & $\hbar\omega(\mathbf{k}) = 2SJn\left(1 - \dfrac{1}{3}\left(\cos(k_xl) + \cos(k_yl) + \cos(k_zl)\right)\right)$ \\ \hline
                \cite{simon2013oxford}             & $\hbar\omega(\mathbf{k}) = 2JSn\left(1 - \dfrac{1}{3}\left(\cos(k_xl) + \cos(k_yl) + \cos(k_zl)\right)\right)$ \\ \hline
                \cite{coey2010magnetism}           & $\hbar\omega(\mathbf{k}) = 2JSn\left(1 - \dfrac{1}{3}\left(\cos(k_xl) + \cos(k_yl) + \cos(k_zl)\right)\right)$ \\ \hline
                \cite{jensen1991rare}              & $\hbar\omega(\mathbf{k}) = Sn2J\left(1 - \dfrac{1}{3}\left(\cos(k_xl) + \cos(k_yl) + \cos(k_zl)\right)\right)$ \\ \hline
            \end{tabular}
        \end{table}


    \section{SpinW}\label{sec:spinw}
        In this section the method from SpinW paper \cite{toth2015linear} and the results plotted by the code \cite{SpinW} itself are discussed.

        First of all, one has to look at the definition of the spin Hamiltonian, which is provided in eq.~(1) of \cite{toth2015linear}:
        \begin{quote}
            We would like to solve the most general magnetic Hamiltonian
            of interacting localized magnetic moments on a periodic lattice
            using LSWT. To accomplish this, a method is necessary that
            can deal with Hamiltonians where the quadratic spin exchange
            interactions are expressed with $3 \times 3$ matrices. In this case
            the exchange energy of two spins will be a matrix product
            $\mathbf{S}_i^T\mathbf{J}\mathbf{S}_j$, where $\mathbf{S}_i$ is a $3 \times 1$ column vector of the spin operators
            $\{S_i^x, S_i^y, S_i^z\}$ of site $i$ and $\mathbf{J}$ is the exchange matrix coupling
            the two sites. $\langle ... \rangle$ Including the external magnetic field and g-tensor, we
            propose to solve the following Hamiltonian:

            \begin{equation}
                H = \sum_{mi,nj} \mathbf{S}_{mi}^T \mathbf{J}_{mi,nj}\mathbf{S}_{nj} 
                + \sum_{mi} \mathbf{S}_{mi}^T \mathbf{A}_{mi}\mathbf{S}_{mi} + \mu_B\mathbf{H}^T\sum_{mi}g_i\mathbf{S}_{mi}.
                \tag{(1)}
            \end{equation}
            
            The indices $m$, $n$ are indexing the crystallographic unit cell
            (running from $1$ to $L$), while $i$ and $j$ label the magnetic atoms inside the unit cell (running from $1$ to $N$), $\mathbf{H}$ is the external
            magnetic field column vector, $\mu_B$ is the Bohr magneton.
        \end{quote}

        This definition includes double counting and is the same as in eq.~\eqref{eq:hh-main} with an opposite sign of the exchange constant. 
        In order to move to the definition of this paper one needs to introduce the following substitution:
        \begin{equation}
            J \rightarrow -J \label{eq:spinw-sub}
        \end{equation}
        In the case of the cubic system $N = 1$. 
        On the way of the solution two vectors are introduced: $\mathbf{u}_j$ and $\mathbf{v}_j$,
        which are defined from the matrix of local rotations $\mathbf{R}^{\prime}_j$. 
        In the case of the ferromagnetic system no rotation is needed and the local rotation matrix and vectors are 
        (index $j$ is dropped because there is only one magnetic site in unit cell in cubic ferromagnetic system)
        \begin{equation}
            \begin{matrix}
                \mathbf{R}_j = 
                \begin{pmatrix}
                    1 & 0 & 0 \\
                    0 & 1 & 0 \\
                    0 & 0 & 1 \\
                \end{pmatrix}; &
                \mathbf{u}_j = 
                \begin{pmatrix}
                    1 \\
                    \iu \\
                    0 \\
                \end{pmatrix}; &
                \mathbf{v}_j = 
                \begin{pmatrix}
                    0 \\
                    0 \\
                    1 \\
                \end{pmatrix}
            \end{matrix}
        \end{equation}
        As the next step in the paper Hamiltonian is written with the collective creation and annihilation operators in a matrix form (eq.(23) in \cite{toth2015linear}):
        \begin{equation}
            H = \sum_{k \in B.Z.}\mathbf{x}^{\dag}(\mathbf{k})\mathbf{h}(\mathbf{k})\mathbf{x}(\mathbf{k})
        \end{equation}
        where $\mathbf{x}(\mathbf{k}) = [b(\mathbf{k}), b^{\dag}(-\mathbf{k})]^T$ for the cubic system. 
        And matrix $\mathbf{h}(\mathbf{k})$ for the cubic system is (definitions in the eqs.~(25)~and~(26)~and(14) of \cite{toth2015linear} are used)
        \begin{equation}
            \mathbf{h}(\mathbf{k}) = 
            \begin{bmatrix}
                \hbar\omega_1(\mathbf{k}) & 0 \\
                0 & \hbar\omega_1(-\mathbf{k})
            \end{bmatrix}
        \end{equation}
        with ($n = 6$)
        \begin{equation}
            \hbar\omega_1(\mathbf{k}) = SJn\left(\dfrac{1}{3}\left(\cos(k_xl) + \cos(k_yl) + \cos(k_zl)\right) - 1\right)
        \end{equation}
        or in the notation of this paper (substitution~\eqref{eq:spinw-sub})
        \begin{equation}
            \hbar\omega_1(\mathbf{k}) = SJn\left(1 - \dfrac{1}{3}\left(\cos(k_xl) + \cos(k_yl) + \cos(k_zl)\right)\right)
        \end{equation}
        which is exactly twice smaller than the result in eq.~\eqref{eq:main-dispersion}.
        For the general case $\mathbf{h}(\mathbf{k})$ requires further diagonalization, which is discussed in the section~$7$ of \cite{toth2015linear}, 
        in the case of a simple cubic system diagonalization procedure is not necessary.
        The final <<diagonalized>> Hamiltonian is 
        \begin{equation}
            H = \sum_{\mathbf{k} \in B.Z.}\left(\hbar\omega_1(\mathbf{k})\hat{b}^{\dag}(\mathbf{k})\hat{b}(\mathbf{k}) + \hbar\omega_1(-\mathbf{k})\hat{b}(\mathbf{-k})\hat{b}^{\dag}(\mathbf{-k})\right)
        \end{equation}

        Since $[\hat{b}(\mathbf{k})\hat{b}^{\dag}(\mathbf{k})] = 1$
        \begin{equation}
            H = \sum_{\mathbf{k} \in B.Z.}\left(\hbar\omega_1(\mathbf{k})\hat{b}^{\dag}(\mathbf{k})\hat{b}(\mathbf{k}) + \hbar\omega_1(-\mathbf{k})\hat{b}^{\dag}(\mathbf{-k})\hat{b}(\mathbf{-k}) + \hbar\omega_1(\mathbf{k})\right)
            \label{eq:spinw-cub}
        \end{equation}
        which does not have the desired form of the Hamiltonian of the collection of independent quasiparticles (magnons)
        \begin{equation}
            \hat{H} = E_0 + \sum_{\mathbf{k}}\hbar\omega(\mathbf{k})\hat{b}^{\dag}(\mathbf{k})\hat{b}(\mathbf{k})\label{eq:desired-form}
        \end{equation}
    

        \begin{figure}[H]
            \centering
            \begin{subfigure}[b]{0.8\textwidth}
                \centering
                \includegraphics[height=7cm]{spinw-cub.png}
            \end{subfigure}
            \hfill
            \caption{Magnon dispersion plotted with SpinW.}
            \label{fig:spinw-cub}
        \end{figure}

        The sum over the Brillouin zone contains the $-\mathbf{k}$ vector for each $\mathbf{k}$ vector, 
        therefore the eq.\eqref{eq:spinw-cub} can be transformed to the desired form of the eq.~\eqref{eq:desired-form} (the constant energy shift is ignored here):
        \begin{equation}
            \begin{aligned}
                H = \sum_{\mathbf{k} \in B.Z.}\left(\hbar\omega_1(\mathbf{k})\hat{b}^{\dag}(\mathbf{k})\hat{b}(\mathbf{k}) 
                + \hbar\omega_1(-\mathbf{k})\hat{b}^{\dag}(\mathbf{-k})\hat{b}(\mathbf{-k})\right)\\ 
                = \sum_{\mathbf{k} \in B.Z.}\left(\hbar(\omega_1(\mathbf{k}) + \omega_1(\mathbf{k}))\hat{b}^{\dag}(\mathbf{k})\hat{b}(\mathbf{k})\right)
            \end{aligned}
        \end{equation}
        Which in the case of the cubic system leads to the same result as in eq.~\eqref{eq:main-dispersion} in the desired form of eq.~\eqref{eq:desired-form}:
        \begin{equation}
            \begin{aligned}
                H = \sum_{\mathbf{k} \in B.Z.}\left(\hbar\omega(\mathbf{k})\hat{b}^{\dag}(\mathbf{k})\hat{b}(\mathbf{k})\right)
            \end{aligned}
        \end{equation}
        where (in notation of SpinW)
        \begin{equation}
            \hbar\omega(\mathbf{k}) = 2\hbar \omega_1(\mathbf{k}) = 2SJn\left(\dfrac{1}{3}\left(\cos(k_xl) + \cos(k_yl) + \cos(k_zl)\right) - 1\right)
        \end{equation}

        However the SpinW code does plot the energies from eq.~\eqref{eq:spinw-cub}, but multiply the $\hbar\omega_1$ from the second term by $-1$. 
        In Fig.~\ref{fig:spinw-cub} the dispersion for the same path as in Fig.~\ref{fig:main-dispersion} is plotted. 
        The picture is produces with the script <<codes/spinw3D.m>>, which is based on the Tutorial 1 from the SpinW website (\url{https://www.spinw.org/tutorials/01tutorial})


    \section{<<Spin Waves and Magnetic Exchange Hamiltonian in CrSBr>>\cite{scheie2022spin}}
        In this section the fitting of the experimental measurements for magnon dispersion of CrSBr are discussed.

        First of all one has to look at the definition of the Hamiltonian in eq.~(1) of \cite{scheie2022spin}:
        \begin{quote}
            \begin{equation}
                H = \sum_{i,j} J_{\langle ij \rangle} \vec{S}_i \cdot \vec{S}_j
                \label{eq:exp-ham}
            \end{equation}
        \end{quote}

        Double counting is present in this definition (I asked Allen Scheie directly and the answer was yes), 
        therefore the Hamiltonian is defined in the same way as in eq.~\eqref{eq:hh-main} 
        with the substitution $J \rightarrow -J$ (and in the same way as in SpinW). 

        There are two processes in the experimental paper, which we are interested in: 
        \begin{enumerate}
            \item Fitting of experimental data to obtain J values.
            \item Plotting of magnon dispersion with SpinW using fitted J values.
        \end{enumerate}
        
        \subsection{Fit of experimental data}
            The codes of the paper are public (\url{doi.org/10.13139/ORNLNCCS/1869252}) and could be examined. 
            Codes of our interest are located in the <<CrSB\_Data\&SpinWaveFits>> folder.
            The script <<FitSpinWaves.ipynb>> fits the spin-wave energies. 
            In this script several functions from the <<pylib.HeisenbergFMSpinWave>> are used for plotting of magnon branches. 
            Those functions are written in the file <<pylib/HeisenbergFMSpinWave.py>>.
            Computation of acoustic and optical branches traces down to the functions <<hw\_ac>> and <<hw\_op>> form this file.
            Those two functions depend on the function <<scriptJs>> and global variables <<Jz>> and <<qzero>>. 
            Finally, there is a piece of code from which one could extract dispersion law used for the fit:
            \begin{python}
                @njit
                def scriptJs(qvec, J):
                    '''create J1 and J2'''
                    
                    # Define the q vectors and the energies
                    qvrlu = qvec*2*np.pi
                    en1 = np.zeros_like(qvec[:,0], dtype=np.complex128)
                    en2 = np.zeros_like(qvec[:,0], dtype=np.complex128)
                    
                    # Loop through all neighbors
                    for i in range(len(dists)):
                        d = dists[i]
                        if d[-2] >= len(J):  #we got to highest J defined
                            break
                        
                        rmr = d[:3] #distance between sites
                        Ji = J[int(d[-2])]  #exchange constant
                        
                        ### Removed factor of 2 compared to Gd.
                        if d[-3] == 1: #It's a site-1 to site-1 exchange
                            en1 += Ji*np.exp(-1j*np.dot(qvrlu,rmr))
                        elif d[-3] == 2: #It's a site-1 to site-2 exchange
                            en2 += Ji*np.exp(-1j*np.dot(qvrlu,rmr))
                        
                    return en1, en2

                qzero = np.array([[0,0,0]], dtype=np.float64)
                Jz = 3/2
                Anisotropy = 0.0

                @njit
                def hw_ac(qvec, J):
                    '''accoustic magnon branch. J is a list of exchange values 
                    from nearest neighbor to further neighbor'''
                    j10, j20 = scriptJs(qzero, J)
                    j1, j2 = scriptJs(qvec, J)
                    return -Jz*(j10 + j20 - j1 - np.abs(j2)) + Anisotropy
                    
                @njit
                def hw_op(qvec, J):
                    '''optical magnon branch. J is a list of exchange values 
                    from nearest neighbor to further neighbor'''
                    j10, j20 = scriptJs(qzero, J)
                    j1, j2 = scriptJs(qvec, J)
                    return -Jz*(j10 + j20 - j1 + np.abs(j2)) + Anisotropy
            \end{python}

            The function <<scriptJs>> prepare $\mathcal{J}(\mathbf{k})$ sum:
            \begin{equation}
                \mathcal{J}(\mathbf{k}) = \sum_{\delta_i} J_i e^{-\iu\mathbf{k}\boldsymbol{\delta}_i} 
            \end{equation}
            Then functions <<hw\_ac>> and <<hw\_op>> compute dispersion law for the 
            2 branches, with the equation (for acoustic branch) ($J_z$ in the code is $S$ here):
            \begin{equation}
                \hbar\omega(\mathbf{k}) = -S (\mathcal{J}(\mathbf{0}) - \mathcal{J}(\mathbf{k}))
                \label{eq:exp-disp}
            \end{equation}
            which is the same as the formula from <<Rare earth magnetism>>\cite{jensen1991rare} (section~\ref{subsec:rem}) with an additional <<$-$>> sign, 
            but the definition of the Hamiltonian in this book is different from the one stated 
            in experimental paper, thus the factor $2$ is missing in the eq.~\eqref{eq:exp-disp}.
            Therefore with respect to the definition of the Hamiltonian in eq.~\eqref{eq:exp-ham} 
            the J values in the experimental paper may be overestimated by the factor of $2$.

        \subsection{Plot with SpinW}
            Plots with SpinW (scripts and input data) are located in <<CrSB\_Data\&SpinWaveFits/SpinW\_code>> folder. 
            The script, which plots the dispersion of CrSBr is <<CSB\_dispersions.m>>. 
            It produces the following pictures for the dispersion:

            \begin{figure}[H]
                \centering
                \begin{subfigure}[b]{0.8\textwidth}
                    \centering
                    \includegraphics[height=14cm]{spinw-exp-before.png}
                \end{subfigure}
                \hfill
                \caption{Magnon dispersion from experimental paper plotted with SpinW.}
                \label{fig:spinw-exp-before}
            \end{figure}

            In order to see the second set of branches as in the example of ferromagnetic cubic system (section~\ref{sec:spinw}) The code is slightly modified:
            line
            \begin{python}
                sw_plotspec(csbSpec,'axLim',[0 50],'mode',3,'dE',2,'colorbar',false,'legend',false)
            \end{python}
            is substituted with
            \begin{python}
                sw_plotspec(csbSpec,'mode',1,'colorbar',false,'legend',false)
            \end{python}
            which produces the following picture:

            \begin{figure}[H]
                \centering
                \begin{subfigure}[b]{0.8\textwidth}
                    \centering
                    \includegraphics[height=14cm]{spinw-exp-after.png}
                \end{subfigure}
                \hfill
                \caption{Magnon dispersion from experimental paper plotted with SpinW (with code modification).}
                \label{fig:spinw-exp-after}
            \end{figure}

            One could see that the magnon dispersion (left column) has two sets of branches, 
            which are equal to each other with one being multiply by $-1$, as it happens with the example of ferromagnetic cubic system.
            In both cases the values of magnon dispersion are the half of the one from the method of this paper and J values from experimental paper. 

            In conclusion: The SpinW plots are underestimated by the factor $1/2$.

        The final conclusion is that since exchange values are overestimated by the factor of $2$ and magnon dispersion 
        from SpinW in underestimated by the factor of $1/2$ the results in the experimental paper are consistent with each other ($2 \cdot 1/2 = 1$), 
        but when one wants to use the J values from experimental paper independently of SpinW code the mismatch of the factor $2$ happens.

    \section{Appendix I}

In this section The magnon dispersion law is derived from the Hamiltonian in eq.~\eqref{eq:hh-main}.
First of all, the Hamiltonian is rewritten with the raising and lowering spin operators:

\begin{equation}
    \hat{S}_i^{\pm} = \hat{S}_i^x \pm \iu \hat{S}_i^y
\end{equation}

\begin{equation}
    \begin{matrix}
        \hat{\mathbf{S}}_i^T \hat{\mathbf{S}}_j = 
        \hat{S}_i^x \hat{S}_j^x + \hat{S}_i^y \hat{S}_j^y + \hat{S}_i^z \hat{S}_j^z; &
        \hat{S}_i^x \hat{S}_j^x + \hat{S}_i^y \hat{S}_j^y = 
        \dfrac{1}{2}\left(\hat{S}_i^+\hat{S}_j^- + \hat{S}_i^-\hat{S}_j^+\right)
    \end{matrix}
\end{equation}

\begin{equation}
    \hat{H} = -J \sum_{ij} \left(\dfrac{1}{2}\left(
        \hat{S}_i^+\hat{S}_j^- + \hat{S}_i^-\hat{S}_j^+\right) + \hat{S}_i^z \hat{S}_j^z\right)
\end{equation}

Since the commutator is 
\begin{equation}
    \left[\hat{S}_i^+\hat{S}_j^-\right] = 2\hat{S}_i^z\delta_{ij}
\end{equation}
and $i\ne j$ in the sum the Hamiltonian becomes:
\begin{equation}
    \hat{H} =-J \sum_{ij} \left(\dfrac{1}{2}\left(
            \hat{S}_j^-\hat{S}_i^+ + \hat{S}_i^-\hat{S}_j^+\right) + \hat{S}_i^z \hat{S}_j^z\right)
\end{equation}

The spin-wave Hamiltonian is obtained with the linearised Holstein–Primakoff formalism.

\begin{equation}
    \begin{matrix}
        \hat{S}_i^+ = \sqrt{2S}\hat{a}_i \\
        \hat{S}_i^- = \sqrt{2S}\hat{a}_i^{\dag} \\
        \hat{S}_i^z = S - \hat{a}_i^{\dag}\hat{a}_i
    \end{matrix}
\end{equation}

\begin{equation}
    \hat{H} = -J \sum_{ij} \left(\dfrac{1}{2}\left(
        2S\hat{a}_j^{\dag}\hat{a}_i + 2S\hat{a}_i^{\dag}\hat{a}_j\right) + 
        \left(S - \hat{a}_i^{\dag}\hat{a}_i\right)\left(S - \hat{a}_j^{\dag}\hat{a}_j\right)\right)
\end{equation}

\begin{equation}
    \hat{H} = E_0 + \hat{H}^{(2)} + \dots 
\end{equation}
\begin{equation}
    E_0 = -JS^2Nn \label{eq:zero-energy}
\end{equation}
\begin{equation}
    \hat{H}^{(2)} = -JS \sum_{ij} \left(\hat{a}_j^{\dag}\hat{a}_i + \hat{a}_i^{\dag}\hat{a}_j - 
    \hat{a}_i^{\dag}\hat{a}_i - \hat{a}_j^{\dag}\hat{a}_j\right)
    \label{eq:quadratic-ham}
\end{equation}
where $N$ is the number of spins in the system, $n$ - number of neighbors for each spin ($6$ in the case of cubic system).
From this point the quadratic part of the Hamiltonian $\hat{H}^{(2)}$ is considered.

The Fourier transform is introduced to move from the local operators $\hat{a}_i^{\dag}$ and $\hat{a}_i$
to the collective creation and annihilation operators $\hat{a}_k^{\dag}$ and $\hat{a}_k$:
\begin{equation}
    \hat{a}_i = \dfrac{1}{\sqrt{N}}\sum_k e^{\iu\mathbf{k}\mathbf{r}_i} \hat{a}_k 
\end{equation}
\begin{equation}
    \hat{a}_i^{\dag} = \dfrac{1}{\sqrt{N}}\sum_k e^{-\iu\mathbf{k}\mathbf{r}_i} \hat{a}_k^{\dag} 
\end{equation}
\begin{equation}
    \dfrac{1}{N}\sum_i e^{\iu(\mathbf{k} - \mathbf{k}^{\prime})\mathbf{r}_i} = \delta_{kk^{\prime}} 
\end{equation}

\begin{equation}
\begin{aligned}
    \hat{H}^{(2)}  = -JS \sum_i\sum_j \dfrac{1}{N}\left[
        \left(\sum_k e^{-\iu\mathbf{k}\mathbf{r}_j} \hat{a}_k^{\dag}\right)
        \left(\sum_{k^{\prime}} e^{\iu\mathbf{k^{\prime}}\mathbf{r}_i} \hat{a}_{k^{\prime}}\right)\right. \\
        +
        \left(\sum_k e^{-\iu\mathbf{k}\mathbf{r}_i} \hat{a}_k^{\dag}\right)
        \left(\sum_{k^{\prime}} e^{\iu\mathbf{k^{\prime}}\mathbf{r}_j} \hat{a}_{k^{\prime}}\right)  \\
        -
        \left(\sum_k e^{-\iu\mathbf{k}\mathbf{r}_i} \hat{a}_k^{\dag}\right)
        \left(\sum_{k^{\prime}} e^{\iu\mathbf{k^{\prime}}\mathbf{r}_i} \hat{a}_{k^{\prime}}\right)  \\
        -
        \left.\left(\sum_k e^{-\iu\mathbf{k}\mathbf{r}_j} \hat{a}_k^{\dag}\right)
        \left(\sum_{k^{\prime}} e^{\iu\mathbf{k^{\prime}}\mathbf{r}_j} \hat{a}_{k^{\prime}}\right) \right]
\end{aligned}
\end{equation}

Since for each $i$ there is the same pattern of neighbors sum over $j$ does not depend on $i$ and it can be moved freely.
Lets define $\boldsymbol{\delta}_j = \mathbf{r}_j - \mathbf{r}_i$ and rewrite the equation:

\begin{equation}
\begin{aligned}
    \hat{H}^{(2)}  = -JS \sum_k\sum_{k^{\prime}}\sum_j \left[
        e^{-\iu\boldsymbol{\delta}_j\mathbf{k}}
        \left(\dfrac{1}{N}\sum_ie^{\iu(\mathbf{k^{\prime}}-\mathbf{k})\mathbf{r}_i}\right) 
        \hat{a}_k^{\dag}\hat{a}_{k^{\prime}}\right. \\
        +
        e^{\iu\boldsymbol{\delta}_j\mathbf{k}}
        \left(\dfrac{1}{N}\sum_ie^{\iu(\mathbf{k^{\prime}}-\mathbf{k})\mathbf{r}_i}\right)
         \hat{a}_k^{\dag}\hat{a}_{k^{\prime}}  \\
        -
        \left(\dfrac{1}{N}\sum_ie^{\iu(\mathbf{k^{\prime}}-\mathbf{k})\mathbf{r}_i}\right)
        \hat{a}_k^{\dag}\hat{a}_{k^{\prime}}  \\
        -
        \left.e^{\iu(\mathbf{k^{\prime}}-\mathbf{k})\boldsymbol{\delta}_j}
        \left(\dfrac{1}{N}\sum_ie^{\iu(\mathbf{k^{\prime}}-\mathbf{k})\mathbf{r}_i}\right)
        \hat{a}_k^{\dag}\hat{a}_{k^{\prime}} \right]
\end{aligned}
\end{equation}
every equation in round parenthesis is equal to $\delta_{kk^{\prime}}$ and the Hamiltonian becomes

\begin{multline}
    \hat{H}^{(2)} = -JS\sum_k\sum_j\left[e^{-\iu\boldsymbol{\delta}_j\mathbf{k}}
    \hat{a}_k^{\dag}\hat{a}_k
    +
    e^{\iu\boldsymbol{\delta}_j\mathbf{k}}
     \hat{a}_k^{\dag}\hat{a}_k 
    -
    \hat{a}_k^{\dag}\hat{a}_k 
    -
    e^{\iu(\mathbf{k}-\mathbf{k})\boldsymbol{\delta}_j}
    \hat{a}_k^{\dag}\hat{a}_k\right] \\
     = 2JS\sum_k\sum_j(1 - \cos(\boldsymbol{\delta}_j\mathbf{k}))\hat{a}_k^{\dag}\hat{a}_k
\end{multline}
$j$ runs from $1$ to $n$, therefore:
\begin{equation}
    \hat{H}^{(2)} = 2JSn\sum_k\left(1 - \dfrac{1}{n}\sum_j
    \cos(\boldsymbol{\delta}_j\mathbf{k})\right)\hat{a}_k^{\dag}\hat{a}_k = 
    \sum_k \hbar\omega(\mathbf{k})\hat{a}_k^{\dag}\hat{a}_k
\end{equation}

For the cubic system ($l$ - length of the lattice vector):
\begin{align}
    \dfrac{1}{n}\sum_j
    e^{\iu\boldsymbol{\delta}_j\mathbf{k}}  = \dfrac{1}{6}\left(\cos(k_xl) + \cos(-k_xl) + \cos(k_yl) + \cos(-k_yl) + \cos(k_zl) + \cos(-k_zl)\right) \\
    = \dfrac{1}{3}\left(\cos(k_x l) + \cos(k_y l) + \cos(k_z l)\right)
\end{align}
and the final formula for the magnon dispersion is
\begin{equation}
    \hbar\omega(\mathbf{k}) = 2JSn\left(1 - 
    \dfrac{1}{3}\left(\cos(k_x l) + \cos(k_y l) + \cos(k_z l)\right)\right)
\end{equation}

    \begingroup
    \let\clearpage\relax
    \section{Appendix II}

\subsection{<<Fundamentals of Magnonics>>\cite{rezende2020fundamentals}}

    In <<Fundamentals of Magnonics>> the derivation of magnon dispersion is done in chapter~$3$~<<Quantum Theory of Spin Waves: Magnons>>.

    The Hamiltonian is defined on page~$72$, equation~\eqref{eq:fom-3.6} as follows:

    \begin{quote}
        \begin{equation}
            H = -g\mu_B\sum_iH_zS_i^z - J\sum_{i, \delta}\vec{S}_i \cdot \vec{S}_{i+\delta}, \label{eq:fom-3.6} \tag{3.6}
        \end{equation}
        where$\vec{S}_i$ is spin angular momentum operator as site $i$, <...> and $\vec{\delta}$ is the vector connecting site $i$ with its nearest neighbors. 
        <...> Notice also that the factor 2 in the exchange energy does not appear explicitly because each pair of spins is counted twice in the sum over lattice sites.
    \end{quote}

    The definition of the Hamiltonian (we ignore the Zeeman term) is the same as in~\eqref{eq:hh-main}: $i + \delta \rightarrow j$ and $\sum_{i, \delta} \rightarrow \sum_{i,j}$, therefore, no conversion is needed for parameters for this textbook.

    Magnon dispersion is provided on page $78$ in equations~\eqref{eq:fom-3.35}~and~\eqref{eq:fom-3.36}

    \begin{quote}
        \begin{equation}
            E_k = A_k = g\mu_B H_z + 2zJS(1 - \gamma_k), \label{eq:fom-3.35} \tag{3.35}
        \end{equation}
        where $\gamma_k$ is the structure factor given by
        \begin{equation}
            \gamma_k = \dfrac{1}{z}\sum_{\vec{\delta}}e^{\iu \vec{k}\cdot\vec{\delta}} \label{eq:fom-3.36} \tag{3.36}
        \end{equation}
    \end{quote}
    where $z$ is the number of neighbors ($n$ in the notation of this paper). $\gamma_k$ for the cubic system is (it is provided on page~$79$ in equation~\eqref{eq:fom-3.37}):
    \begin{quote}
        \begin{equation}
            \gamma_k = \dfrac{1}{3}(\cos(k_xa) + \cos(k_ya) + \cos(k_za)) \label{eq:fom-3.37} \tag{3.37}
        \end{equation}
    \end{quote}
    where $a$ is a lattice parameter ($l$ in the notation of this paper). The final equation from the \cite{rezende2020fundamentals} in the notation of this paper is
    \begin{equation}
        \hbar\omega(\mathbf{k}) = 2nJS\left(1 - \dfrac{1}{3}\left(\cos(k_xl) + \cos(k_yl) + \cos(k_zl)\right)\right)
        \label{eq:rezende}
    \end{equation}

    This equation is the same as equation~\eqref{eq:main-dispersion}.

\subsection{<<Magnetism in condensed matter>>\cite{blundell2003magnetism}}
    The derivation of magnon dispersion for the ferromagnetic 1D chain is discussed in the section~$6.6.6$~<<Magnons>>.

    The definition of the Hamiltonian is provided on page~$122$ in equations~\eqref{eq:micm-6.9}~and~\eqref{eq:micm-6.10}
    \begin{quote}
        (1) We begin with a semiclassical derivation of the spin wave dispersion.
        First, recall the Hamiltonian for the Heisenberg model,
        \begin{equation}
            \hatH = -\sum_{\langle ij\rangle} J\hat{\mathbf{S}}_i \cdot \hat{\mathbf{S}}_j \label{eq:micm-6.9} \tag{6.9}
        \end{equation}
        (which is eqn. $6.4$) In a one-dimensional chain each spin has two neighbours, so the Hamiltonian reduces to
        \begin{equation}
            \hatH = -2J\sum_{i} \hat{\mathbf{S}}_i \cdot \hat{\mathbf{S}}_{i+1} \label{eq:micm-6.10} \tag{6.10}
        \end{equation}
    \end{quote}
    with the comment to the equation~($6.4$) on the page~$116$ being
    \begin{quote}
        where the constant $J$ is the exchange integral and the symbol $\langle ij \rangle$ below the $\sum$ denotes a sum over nearest neighbours. 
        The spins $\mathbf{S}_i$ are treated as three-dimensional vectors ...
    \end{quote}

    The definition of the Heisenberg model is found for the first time in the section~$4.2.1$ on the page~$76$ in equations~\eqref{eq:micm-4.7}~and~\eqref{eq:micm-4.8}:
    \begin{quote}
        This motivates the Hamiltonian of the Heisenberg model:
        \begin{equation}
            \hatH = -\sum_{ij}J_{ij}\mathbf{S}_i \cdot \mathbf{S}_j, \label{eq:micm-4.7} \tag{4.7}
        \end{equation}
        where $J_{ij}$ is the exchange constant between the $i^{\text{th}}$ and $j^{\text{th}}$ spins. 
        The factor of 2 is omitted because the summation includes each pair of spins twice. 
        Another way of writing eqn $4.7$ is 
        \begin{equation}
            \hatH = -2\sum_{i>j}J_{ij}\mathbf{S}_i \cdot \mathbf{S}_j, \label{eq:micm-4.8} \tag{4.8}
        \end{equation}
        where the $i > j$ avoids the <<double-counting>> and hence the factor of two returns. 
        Often it is possible to take $J_{ij}$ to be equal to a constant $J$ for nearest neighbours spins and to be $0$ otherwise.
    \end{quote}
    The equation~\eqref{eq:micm-6.9} corresponds to the definition in equation~\eqref{eq:micm-4.7} 
    and the equation~\eqref{eq:micm-6.10} corresponds to the definition in equation~\eqref{eq:micm-4.8}. 
    The definition in equation~\eqref{eq:micm-4.7} is the same as in equation~\eqref{eq:hh-main}, 
    therefore, no conversion of parameters is needed for this textbook.

    The Hamiltonian is solved specifically for the ferromagnetic 1D chain and not for the 3D cubic system with the final result (equation~\ref{eq:micm-6.20} on page~$123$ and equation~\ref{eq:micm-6.25} on page~$124$)
    \begin{quote}
        \begin{equation}
            \hbar\omega = 4JS(1 - \cos(qa)), \label{eq:micm-6.20} \tag{6.20}
        \end{equation}
        \begin{equation}
           E(q) = -2NS^2J + 4JS(1 - \cos(qa)), \label{eq:micm-6.25} \tag{6.25}
        \end{equation}
    \end{quote}
    which matches with the equations~\eqref{eq:zero-energy}~and~\eqref{eq:main-dispersion} if $n = 2$ is used and 1D-chain instead of 3D cubic system is considered. 
    Magnon dispersion from equation~\eqref{eq:micm-6.20} is plotted in the book on page~$123$ in figure~$6.12$ (Fig.~\ref{fig:micm-6.12}). 
    Path from $0$ to $\pi / a$ corresponds to the $\Gamma$-X path in Fig.~\ref{fig:main-dispersion}. If the parameters from this paper are substitute into the eq.~\eqref{eq:micm-6.20} then those two graphs are be exactly the same.
    \begin{quote}
        \begin{figure}[H]
            \centering
            \begin{subfigure}[b]{0.5\textwidth}
                \centering
                \includegraphics[height=6cm]{micm-6.12.png}
            \end{subfigure}
            \hfill
            \caption{Magnon dispersion plot from <<Magnetism in condensed matter>>.}
            \label{fig:micm-6.12}
        \end{figure}
    \end{quote}
    For the cubic system eq.~\ref{eq:micm-6.20} will look like:
    \begin{equation}
        \hbar\omega = 12JS(1 - \dfrac{1}{3}(\cos(q_xa) + \cos(q_ya) + \cos(q_za)))
    \end{equation}
    If it is to be rewritten with the notation of this paper it will look like
    \begin{equation}
        \hbar\omega = 2nJS(1 - \dfrac{1}{3}(\cos(k_xl) + \cos(k_yl) + \cos(k_zl)))
    \end{equation}

\subsection{<<Magnetisation oscillations and waves>>\cite{gurevich1996magnetization}}
    The derivation of magnon dispersion for the ferromagnet is discussed in the section~$7.4$~<<Elements of microscopic spin-wave theory>>.

    The definition of the Hamiltonian is provided on page~$205$ in equations~\eqref{eq:moaw-7.82}~and~\eqref{eq:moaw-7.82}

    \begin{quote}
        \begin{equation}
            \hatH = \gamma\hbar\sum_{f}\hat{S}_f^z - \sum_f\sum_{f^{\prime} \ne f} I_{ff^{\prime}}\mathbf{S}_f\mathbf{S}_{f^{\prime}} \label{eq:moaw-7.82} \tag{7.82}
        \end{equation}
        where $\mathbf{S}_f\mathbf{S}_{f^{\prime}} = \hat{S}_f^x\hat{S}_{f^{\prime}}^x + \hat{S}_f^y\hat{S}_{f^{\prime}}^y + \hat{S}_f^z\hat{S}_{f^{\prime}}^z$.
    \end{quote}
    The double counting is present in this Hamiltonian, thus, it is the same definition as in eq.~\eqref{eq:hh-main} of this paper with the following notation change:
    \begin{equation}
        \begin{matrix}
            f \rightarrow i, & 
            f^{\prime} \ne f \rightarrow j, & 
            I_{ff^{\prime}} \rightarrow J, & 
            \mathbf{S}_f \rightarrow \mathbf{S}_i, &
            \mathbf{S}_{f^{\prime}} \rightarrow \mathbf{S}_j
        \end{matrix}
    \end{equation}

    The dispersion law is provided in equation~\eqref{eq:moaw-7.99} on page~$209$
    \begin{quote}
        where $r_g = r_f - r_{f^{\prime}}$, $I_g \equiv I_{ff^{\prime}}$, and the last sum is over all lattice points except one, the initial. 
        The Hamiltonian ($7.98$) has the desired form of ($7.84$), and
        \begin{equation}
            \varepsilon_k(k) = \gamma\hbar H + 2S \sum_g [1 - exp(\iu \mathbf{k}\mathbf{r}_g)]I_g. \label{eq:moaw-7.99} \tag{7.99}
        \end{equation}
    \end{quote}

    For the cubic ferromagnet the textbook provides the figure~7.13 (Fig.~\ref{fig:moaw-7.13})
    \begin{quote}
        \begin{figure}[H]
            \centering
            \begin{subfigure}[b]{0.49\textwidth}
                \centering
                \includegraphics[height=6cm]{moaw-7.13.png}
                \caption{Original plot}
                \label{fig:moaw-7.13-original}
            \end{subfigure}
            \hfill
            \begin{subfigure}[b]{0.49\textwidth}
                \centering
                \includegraphics[height=6cm]{custom-moaw.pdf}
                \caption{Same plot with use of eq.~\eqref{eq:main-dispersion}}
                \label{fig:moaw-7.13-custom}
            \end{subfigure}
            \hfill
            \caption{Magnon dispersion plot from <<Magnetisation oscillations and waves>>.}
            \label{fig:moaw-7.13}
        \end{figure}
    \end{quote}
    In this picture curve~$\langle 100\rangle$ (from $0$ to $\pi$) corresponds to the path $\Gamma$-Y, 
    curve~$\langle 110\rangle$ (from $0$ to $\pi\sqrt{2}$) to the path $\Gamma$-M and
    curve~$\langle 111\rangle$ (from $0$ to $\pi\sqrt{3}$) to the path $\Gamma$-R 
    in the Fig.~\ref{fig:main-dispersion}. In Fig.~\ref{fig:moaw-7.13-custom} the same graph is plotted by using the equation for magnon dispersion from this paper.

    The dispersion law from eq.~\ref{eq:moaw-7.99} for the cubic system will be
    \begin{equation}
        \hbar\omega(\mathbf{k}) = 2SIn\left(1 - \dfrac{1}{3}\left(\cos(k_xr_x) + \cos(k_yr_y) + \cos(k_zr_z)\right)\right)
    \end{equation}
    where $g$ varies from $1$ to $6$ $r_g \in [r_x, -r_x, r_y, -r_y, r_z, -r_z]$ and $I_g = I$ for each $g$. 

    In the notation of this paper the dispersion law becomes
    \begin{equation}
        \hbar\omega(\mathbf{k}) = 2SJn\left(1 - \dfrac{1}{3}\left(\cos(k_xl) + \cos(k_yl) + \cos(k_zl)\right)\right)
    \end{equation}

\subsection{<<The Oxford Solid State Basics>>\cite{simon2013oxford}}

\subsection{<<Magnetism and magnetic materials>>\cite{coey2010magnetism}}

\subsection{<<Rare earth magnetism>>\cite{jensen1991rare}}




    \endgroup

    \bibliographystyle{plain} 
    \bibliography{refs.bib} 
    % \section*{References}
    % \printbibliography[title = {\vspace{-2em}}]



\end{document}
